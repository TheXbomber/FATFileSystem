\documentclass[12pt,a4paper,titlepage]{report}
\usepackage[utf8]{inputenc}
\usepackage[T1]{fontenc}
%\usepackage{amsmath}
%\usepackage{amssymb}
\usepackage{graphicx}
\usepackage[left=3.00cm, right=3.00cm, top=3.00cm, bottom=3.00cm]{geometry}
\usepackage[italian]{babel}
\usepackage[hidelinks]{hyperref}
\usepackage{fancyhdr}
\usepackage{titlesec}
\usepackage{float}
\usepackage{wrapfig}
%\usepackage{amsthm}
%\newtheorem*{theorem*}{Teorema}
\usepackage{listings}
\usepackage[dvipsnames]{xcolor}
\usepackage{fancyvrb}

\title{\textbf{FAT File System}\\Progetto per il corso di Sistemi Operativi\\2022}
\author{Alessandro Rocchi}
\date{}

\pagestyle{fancy}
\assignpagestyle{\chapter}{fancy}
\fancyhf{}
%\lfoot{Indice}
\lfoot{FAT File System}
\rfoot{\thepage}
\renewcommand{\headrulewidth}{1pt}
\renewcommand{\footrulewidth}{1pt}
\setcounter{tocdepth}{1}
\hypersetup{linktoc=all}

\lstdefinestyle{mystyle}{   
	commentstyle=\color{Green},
	keywordstyle=\color{blue},
	numberstyle=\tiny\color{Black},
	stringstyle=\color{YellowOrange},
	basicstyle=\ttfamily\footnotesize,
	breakatwhitespace=false,         
	breaklines=true,                                    
	keepspaces=true,                 
	%numbers=left,                    
	%numbersep=10pt,                  
	showspaces=false,                
	showstringspaces=false,
	showtabs=false,                  
	%tabsize=2
}
\lstset{style=mystyle}

\begin{document}
	\maketitle
	%\tableofcontents
	\chapter{Scopo e funzionalità}
	\section{Scopo}
	Lo scopo del programma è fornire alcune tra le funzionalità offerte da un file system attraverso un'interfaccia a riga di comando.\\
	I contenuti del file system sono salvati su un file \verb|my_disk.img| che viene mappato in memoria ad ogni esecuzione del programma simulando un disco.\\
	
	\section{Istruzioni per l'esecuzione}
	Per compilare il programma è necessario spostarsi nella cartella \verb|src| ed eseguire il comando \verb|make|.\\
	Per lanciare il programma usare il comando \verb|./shell|.\\\\
	Per formattare il disco eseguire il comando \verb|make clean|.\\
	
	\section{Funzionalità}
	Una lista delle funzionalità può essere ottenuta digitando il comando \verb|help| all'interno della shell.\\
	Tramite la shell è possibile:
	\begin{itemize}
		\item creare ed eliminare file e directory;
		\item leggere e scrivere file;
		\item cambiare la posizione corrente in un file;
		\item visualizzare e cambiare la directory corrente;
		\item ottenere una lista dei contenuti della directory corrente.
	\end{itemize}
	
	\chapter{Struttura}
	
	\section{Disco}
	Il disco è diviso in tre sezioni:
	\begin{itemize}
		\item la prima sezione contiene delle variabili "globali" a tutto il disco;
		\item la seconda sezione contiene la struttura rappresentante la FAT (\verb|Fat|);
		\item la terza sezione contiene il buffer dove vengono effettivamente memorizzati i dati (\verb|data|).
	\end{itemize}
	\lstinputlisting[language=c]{code/disk.h}\quad\\

	\section{FAT}
	La struttura \verb|Fat| contiene:
	\begin{itemize}
		\item una variabile che tiene il conto dei blocchi liberi nella FAT;
		\item un array contenente i blocchi della FAT.
	\end{itemize}
	\lstinputlisting[language=c]{code/fat.h}\quad\\\\
	Un blocco della FAT (\verb|FatEntry|) contiene:
	\begin{itemize}
		\item l'indice al blocco del disco relativo;
		\item l'indice al prossimo blocco della FAT;
	\end{itemize}
	\lstinputlisting[language=c]{code/fat_entry.h}\quad\\
	
	\section{File}
	Un file è costituito da un header (\verb|FileHead|) e uno o più blocchi dati (\verb|File|).\\\\
	L'header contiene diverse informazioni sul file, come il nome, dimensione, posizione attuale, ecc...
	\lstinputlisting[language=c]{code/file_head.h}\quad\\\\
	Un blocco dati contiene:
	\begin{itemize}
		\item il suoi indice nel disco;
		\item il numero di bytes liberi nel file;
		\item il buffer in cui memorizzare i dati.
	\end{itemize}
	\lstinputlisting[language=c]{code/file.h}
	
	\section{Directory}
	Una directory è rappresentata da una struttura \verb|Dir|, che contiene diverse informazioni sulla directory, tra cui un array di indici dei file e delle sottodirectory che contiene.
	\lstinputlisting[language=c]{code/dir.h}\quad\\
	
\end{document}